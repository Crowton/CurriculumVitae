\cvevent{}{Random Access on Narrow Decision Diagrams in External Memory}
{2024 \textit{Accepted at SPIN'24}}{}

{\small

Authors: S. C. S{\o}lvsten, C. M. Rysgaard, and J. Pol

% \vspace{4pt}

% Abstract: The external memory BDD package Adiar can manipulate Binary Decision Diagrams (BDDs) larger than
% the RAM of the machine. To do so, it uses one or more priority queues to defer processing each
% recursion until the relevant nodes are encountered in a sequential scan.

% We outline how to improve the performance of Adiar's algorithms if the BDD width of one of its
% inputs is small enough to fit into main memory. In this case, one of the algorithms' priority
% queues can entirely be replaced with (levelised) random access to the nodes of the narrow BDD.
% This preserves the I/O efficiency of the original algorithm, is applicable to other types of
% decision diagrams, and significantly improves performance for many larger BDD computations.

}
