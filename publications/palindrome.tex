\cvevent{}{On Finding Longest Palindromic Subsequences\newline using Longest Common Subsequences}
{2024 \textit{Under review}}{}

{\small

Authors: G. S. Brodal, R. Fagerberg, and C. M. Rysgaard

% \vspace{4pt}

% Abstract: Two standard textbook problems illustrating dynamic programming is to
% find the longest common subsequence (LCS) between two strings, and to find the
% longest palindromic subsequence (LPS) of a string. A popular claim is that the
% longest palindromic subsequence in a string can be computed as the longest
% common subsequence between the string and the reversed string. We prove that the
% correctness of this claim depends on how the longest common subsequence is
% computed. In particular, we prove that the classical dynamic programming
% solution by Wagner and Fischer %[JACM\;1974] 
% for finding an LCS in fact does find an LPS, while a slightly different LCS 
% backtracking strategy makes the algorithm fail to always report a palindrome.

}
