%%%%%%%%%%%%%%%%%
% This is an sample CV template created using altacv.cls
% (v1.1.4, 27 July 2018) written by LianTze Lim (liantze@gmail.com). Now compiles with pdfLaTeX, XeLaTeX and LuaLaTeX.
%
%% It may be distributed and/or modified under the
%% conditions of the LaTeX Project Public License, either version 1.3
%% of this license or (at your option) any later version.
%% The latest version of this license is in
%%    http://www.latex-project.org/lppl.txt
%% and version 1.3 or later is part of all distributions of LaTeX
%% version 2003/12/01 or later.
%%%%%%%%%%%%%%%%

%% If you need to pass whatever options to xcolor
\PassOptionsToPackage{dvipsnames}{xcolor}

%% If you are using \orcid or academicons
%% icons, make sure you have the academicons
%% option here, and compile with XeLaTeX
%% or LuaLaTeX.
% \documentclass[10pt,a4paper,academicons]{altacv}

%% Use the "normalphoto" option if you want a normal photo instead of cropped to a circle
% \documentclass[10pt,a4paper,normalphoto]{altacv}

\documentclass[10pt,a4paper,withhyper]{altacv}

%% AltaCV uses the fontawesome and academicon fonts
%% and packages.
%% See texdoc.net/pkg/fontawecome and http://texdoc.net/pkg/academicons for full list of symbols.
%%
%% Compile with LuaLaTeX for best results. If you
%% want to use XeLaTeX, you may need to install
%% Academicons.ttf in your operating system's font
%% folder.


% Change the page layout if you need to
\geometry{left=1cm,right=9cm,marginparwidth=6.8cm,marginparsep=1.2cm,top=1cm,bottom=1.25cm}

% Change the font if you want to.

% If using pdflatex:
\usepackage[utf8]{inputenc}
\usepackage[T1]{fontenc}
\usepackage[default]{lato}

% If using xelatex or lualatex:
% \setmainfont{Lato}

% Change the colours if you want to
\definecolor{SlateGrey}{HTML}{2E2E2E}
\definecolor{LightGrey}{HTML}{666666}
\definecolor{DarkPastelRed}{HTML}{450808}
\definecolor{PastelRed}{HTML}{8F0D0D}
\definecolor{GoldenEarth}{HTML}{E7D192}
\colorlet{name}{black}
\colorlet{tagline}{PastelRed}
\colorlet{heading}{DarkPastelRed}
\colorlet{headingrule}{GoldenEarth}
\colorlet{subheading}{PastelRed}
\colorlet{accent}{PastelRed}
\colorlet{emphasis}{SlateGrey}
\colorlet{body}{LightGrey}

% Change some fonts, if necessary
\renewcommand{\namefont}{\Huge\rmfamily\bfseries}
\renewcommand{\personalinfofont}{\footnotesize}
\renewcommand{\cvsectionfont}{\LARGE\rmfamily\bfseries}
\renewcommand{\cvsubsectionfont}{\large\bfseries}

% Change the bullets for itemize and rating marker
% for \cvskill if you want to
\renewcommand{\itemmarker}{{\small\textbullet}}
\renewcommand{\ratingmarker}{\faCircle}

%% Publications
\usepackage[backend=biber,style=ieee,sorting=ydnt,defernumbers=true]{biblatex}
%% For removing numbering entirely when using a numeric style
\setlength{\bibhang}{1.25em}
\DeclareFieldFormat{labelnumberwidth}{\makebox[\bibhang][l]{\itemmarker}}
\setlength{\biblabelsep}{0pt}
\defbibheading{pubtype}{\cvsubsection{#1}}
\renewcommand{\bibsetup}{\vspace*{-\baselineskip}}
\AtEveryBibitem{%
  \iffieldundef{doi}{}{\clearfield{url}}%
}
\addbibresource{publications.bib}

% Chance font size
\renewcommand*{\bibfont}{\normalfont\small}

% Remove editors from publications
\DeclareSourcemap{
  \maps[datatype=bibtex, overwrite]{
    \map{
      \step[fieldset=editor, null]
    }
  }
}

%% Links
\usepackage{url}                 % \url{link}

%% Fancy enumeration
\usepackage{enumitem}

\usepackage{tket}

% \usepackage{amsfonts}
% \usepackage{mathtools}
\newcommand{\lrParents}[1]{\left( #1 \right)}
\newcommand{\Oh}[1]{\mathcal{O}\!\lrParents{#1}}

\usepackage{ragged2e}
\renewcommand{\raggedright}{\RaggedRight}
% \renewcommand{\raggedright}{}


\begin{document}

% +----------------------------------------------------------------------------+
% |                                P A G E   I                                 |
% +----------------------------------------------------------------------------+

% Header
\name{Casper Moldrup Rysgaard}
\tagline{Computer Science Ph.D. in Algorithms and Data structures from Aarhus University}
\photo{2.8cm}{portrait}

\personalinfo{% Add custom own with \printinfo{symbol}{detail}
  \email{casperrysgaard@gmail.com}
  \location{Aarhus, Denmark}
  %
  \linkedin{/casper-rysgaard}
  \github{Crowton}
  %
  \orcid{0000-0002-3989-123X}
}

%% Make the header extend all the way to the right
\begin{fullwidth}
  \makecvheader
\end{fullwidth}

% ----------------------------------------
% Side bar
% ----------------------------------------
\marginpar{
  % Magic space
  \vspace{1pt}

  \cvsection{Education}
  \cvevent{Ph.D. in Computer Science}{Aarhus University, Denmark}{2021 -- 2025}{}
  \cvevent{MSc in Computer Science}{Aarhus University, Denmark}{2020 -- 2023}{}
  \cvevent{BSc in Computer Science}{Aarhus University, Denmark}{2017 -- 2020}{}

  
  \cvsection{Teaching}
  \cvevent{Teaching Assistant}{Algorithms and Data Structures}{2018, 2019, 2020, 2021, 2022, 2023, 2024}{}
  \cvevent{}{Introduction to Programming with Scientific Applications (Python)}{2019, 2022, 2023}{}
  \cvevent{}{Programming Languages}{2020, 2021}{}

  
  \cvsection{Volunteer Work}
  { \renewcommand{\ttdefault}{pcr}
    \cvevent{}{Chairman, \TKET}{2022 -- 2023}{}
    { \small
      \TKET is a party- and lecture organisation for students of Computer Science, Math, Physics, and more.
    }
  }

  \cvevent{}{Board Member, DSAU}{2019 -- 2022}{}
  { \small
    DSAU is a Computer Science student organisation hosting social and skill learning events, in collaboration with various companies.
  }

  \cvevent{}{Bartender, Fredagscaféen}{2018 -- 2025}{}
  { \small
    The Computer Science's friday bar at Aarhus University.
  }

  \cvevent{}{Kitchen Responsible, Regnecentralen}{2018 -- 2025}{}
  { \small
    A kitchen and social hub for students.
  }

  \cvevent{}{Tutor, Mat/Fys-Tutorgruppen}{2018 -- 2022, 2025}{}
  { \small
    The tutors welcome the new students to the university.
  }

}

% ----------------------------------------
% Main Page
% ----------------------------------------

\cvsection{About Me}
{½\small
  I am a recent Ph.D. graduate from Aarhus University, supervised by Gerth Stølting Brodal.
  My focus has been on Algorithms and Data structures, with a further focus on I/O efficiency and persistence.

  \medskip
  During my time at Aarhus University, I have spent many years as a Teaching Assistant (TA), with many of these as a lead TA.
  I have, as part of this work, been the exam contact person a few times for written exams, where I handled error clarifications and collecting the repsonses.
  Further, I have helped proofreading written exams sets for both errors and clarification issues. 
  This gave me a lot of usefull insight into how exams are executed, and how the evaluation process works.
  Additionally, my work with TA'ing have given me knowledge on how to grade handins, and how to ask questions in a manner understandable to the students.
}


% Magic space
\vspace{11pt}

% Publications
\cvsection{Publications}
\mynames{Rysgaard/Casper\bibnamedelima Moldrup}
\nocite{*}
{ \footnotesize
  \printbibliography[heading=none]
}

\end{document}
